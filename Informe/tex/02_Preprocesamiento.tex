\newpage

\section{Preprocesamiento}
\label{sec:preprocesamiento}

En esta sección se detalla el tratamiento inicial realizado a los datos para prepararlos para el modelado, replicando el flujo implementado en el notebook \texttt{Obligatorio\_ML.ipynb}.

\subsection{Gestión de valores faltantes}

A partir del análisis de los datos (ver Tabla \ref{tab:dataset-missing}), se aplicaron las siguientes estrategias para manejar los valores nulos:

\begin{itemize}
    \item \textbf{reviews\_per\_month}: Se imputaron los valores nulos con 0, asumiendo que la ausencia de reseñas indica que no hubo actividad en ese aspecto.
    \item \textbf{last\_review}: Se convirtió a formato fecha y se utilizó para crear una nueva característica temporal (\texttt{days\_since\_last\_review}). Los valores nulos en esta nueva característica fueron imputados con el valor máximo observado, interpretando que la falta de reseñas recientes sugiere inactividad o antigüedad mayor al máximo registrado.
    \item \textbf{never\_reviewed}: Se creó una variable binaria para indicar explícitamente si un alojamiento nunca ha tenido reseñas (derivado de los nulos en \texttt{last\_review}).
    \item \textbf{name}: Se imputaron los valores nulos con cadena vacía para permitir el procesamiento de texto posterior.
    \item \textbf{host\_id}: Esta columna fue eliminada por ser un identificador sin valor predictivo directo.
    \item \textbf{host\_name}: Análogo a \texttt{host\_id}.
    \item \textbf{id}: Análogo a \texttt{host\_id}. 
\end{itemize}

En resumen se realizaron las siguientes acciones:

\begin{table}[H]
    \centering
    \begin{tabular}{lc}
        \toprule
        Columna                      & Acción aplicada                     \\
        \midrule
        \texttt{reviews\_per\_month} & Reemplazar \text{NaN} por 0                    \\
        \texttt{last\_review}        & Se reemplaza por \texttt{days\_since\_last\_review} y \texttt{never\_reviewed} \\
        \texttt{name}                & Imputación de nulos (luego procesado en Feature Eng.) \\
        \texttt{host\_id}            & Eliminación                        \\
        \texttt{host\_name}          & Eliminación                        \\
        \texttt{id}                  & Eliminación                        \\
        \bottomrule
    \end{tabular}
    \caption{Resumen de acciones aplicadas para gestionar valores faltantes y columnas irrelevantes.}
\end{table}


\subsection{Gestión de outliers}


\begin{table}[H]
    \centering
    \setlength{\tabcolsep}{3pt}
    \begin{scriptsize}
    \begin{adjustbox}{width=\textwidth}
    \begin{tabular}{lrrrrrrr}
        \toprule
        & \texttt{price} & \texttt{minimum\_\allowbreak nights} & \texttt{number\_\allowbreak of\_\allowbreak reviews} & \texttt{reviews\_\allowbreak per\_\allowbreak month} & \texttt{calculated\_\allowbreak host\_\allowbreak listings\_\allowbreak count} & \texttt{availability\_\allowbreak 365} & \texttt{days\_\allowbreak since\_\allowbreak last\_\allowbreak review} \\
        \midrule
        count & 16013 & 16013 & 16013 & 16013 & 16013 & 16013 & 16013 \\
        mean & 3286.097 & 5.5245 & 15.275 & 0.9 & 6.3838 & 202.4305 & 4705.0473 \\
        std & 13182.2029 & 19.9097 & 30.7468 & 1.3081 & 14.9455 & 134.19 & 1302.0118 \\
        min & 120 & 1 & 0 & 0 & 1 & 0 & 2190 \\
        0\% & 120 & 1 & 0 & 0 & 1 & 0 & 2190 \\
        15\% & 897 & 1 & 0 & 0 & 1 & 36 & 2435 \\
        25\% & 1256 & 1 & 0 & 0 & 1 & 83 & 5449 \\
        50\% & 2033 & 3 & 4 & 0.33 & 1 & 180 & 5449 \\
        75\% & 2990 & 4 & 16 & 1.25 & 4 & 344 & 5449 \\
        90\% & 5082 & 7 & 43 & 2.68 & 14 & 364 & 5449 \\
        95\% & 7175 & 16 & 71 & 3.67 & 33 & 365 & 5449 \\
        99\% & 19134 & 60 & 154 & 5.82 & 87 & 365 & 5449 \\
        max & 597865 & 1125 & 451 & 15 & 105 & 365 & 5449 \\
        variance & 173770472.1478 & 396.3954 & 945.3684 & 1.7112 & 223.3694 & 18006.951 & 1695234.6504 \\
        
        \bottomrule
    \end{tabular}
    \end{adjustbox}
    \end{scriptsize}
    \caption{ Estadísticos descriptivos de las variables numéricas (df\_no\_outliers). }
    \label{tab:df-no-outliers-desc}
\end{table}

La Tabla \ref{tab:df-no-outliers-desc} resume los estadísticos descriptivos antes de aplicar el recorte con IQR.

En base al análisis exploratorio, se identificaron varios outliers en columnas numéricas clave que podrían afectar negativamente el rendimiento de los modelos. Estos son:

\begin{itemize}
    \item \textbf{Precios y estadías muy largas} (\texttt{price}, \texttt{minimum\_nights}) que distorsionan la media y la varianza.
    \item \textbf{Actividad de reseñas atípica} (\texttt{number\_of\_reviews}, \texttt{reviews\_per\_month}) con colas largas.
    \item \textbf{Disponibilidad y múltiples listings} (\texttt{availability\_\allowbreak365}, \texttt{calculated\_\allowbreak host\_\allowbreak listings\_\allowbreak count}) con valores extremos.
    \item \textbf{Antigüedad de reseñas} (\texttt{days\_\allowbreak since\_\allowbreak last\_\allowbreak review}) con grandes diferencias entre anuncios activos e inactivos.
\end{itemize}

\begin{table}[H]
    \centering
    \small
    \begin{tabular}{p{2.5cm} p{3.4cm} p{3.3cm} p{4cm}}
        \toprule
        Concepto & Qué hace & Detecta/Modifica & Ventaja \\
        \midrule
        Winsorize & Reemplaza extremos por percentiles & Modifica outliers & Reduce impacto sin eliminar datos \\
        IQR & $Q3 - Q1$ & Base para detectar outliers & Medida robusta de dispersión \\
        \bottomrule
    \end{tabular}
    \caption{Comparativa de enfoques para detectar/tratar outliers. Se eligió IQR con \texttt{clip} por robustez y simplicidad.}
    \label{tab:outliers-metodos}
\end{table}


Se opto por aplicar el método basado en el rango intercuartílico (IQR) combinado con la función \texttt{clip} para limitar los valores fuera de los límites definidos por $Q1 - 1.5 \times IQR$ y $Q3 + 1.5 \times IQR$. Esto se debe a varias razones:

\begin{itemize}
    \item El IQR es robusto en distribuciones sesgadas y no asume normalidad.
    \item \texttt{clip} evita eliminar filas: se preserva el tamaño muestral para Cross-Validation.
    \item Se integra de forma simple y reproducible en el pipeline de \texttt{scikit-learn} del notebook \texttt{Obligatorio\_ML.ipynb}.
\end{itemize}


Para ver otras gráficas utilizadas para el análisis de outliers, dirigirse a la sección del apéndice \ref{appendix:outliers}.

\subsection{Transformación de variables}

Por otro lado, se aplicó una  transformación logarítmica (\lstinline[style=python]{np.log1p}) a la variable objetivo \texttt{price}. Esto se realizó para estabilizar la varianza y acercar la distribución de la variable a una normal, lo cual beneficia el rendimiento de los modelos de regresión lineal y otros algoritmos sensibles a la escala y distribución de la variable objetivo.

Las métricas reportadas en las secciones siguientes corresponden a la escala logarítmica; al generar predicciones finales se revirtió la transformación con \lstinline[style=python]{np.expm1}.

\subsection{Resultado Final}

En la siguiente tabla (Tabla \ref{tab:df-iqr-desc-log}) se presentan los estadísticos descriptivos de las variables numéricas tras aplicar el preprocesamiento descrito, incluyendo la gestión de valores faltantes y el tratamiento de outliers. Se observa una reducción significativa en la varianza y los valores máximos, indicando que el preprocesamiento ha mitigado el impacto de los outliers en el conjunto de datos.


\begin{table}[H]
    \centering
    \setlength{\tabcolsep}{3pt}
    \begin{scriptsize}
    \begin{adjustbox}{width=\textwidth}
    \begin{tabular}{lrrrrrrr}
        \toprule
        & \texttt{price\_\allowbreak log} & \texttt{minimum\_\allowbreak nights\_\allowbreak log} & \texttt{number\_\allowbreak of\_\allowbreak reviews\_\allowbreak log} & \texttt{reviews\_\allowbreak per\_\allowbreak month} & \texttt{availability\_\allowbreak 365\_\allowbreak log} & \texttt{calculated\_\allowbreak host\_\allowbreak listings\_\allowbreak count\_\allowbreak log} & \texttt{days\_\allowbreak since\_\allowbreak last\_\allowbreak review} \\
        \midrule
        
        count & 16013 & 16013 & 16013 & 16013 & 16013 & 16013 & 16013 \\
        
        mean & 7.5995 & 1.3607 & 1.6827 & 0.8015 & 4.8888 & 1.2401 & 5450 \\
        
        std & 0.734 & 0.6079 & 1.4712 & 0.9998 & 1.1654 & 0.7522 & 0 \\
        
        min & 5.8362 & 0.6931 & 0 & 0 & 2.3117 & 0.6931 & 5450 \\
        
        0\% & 5.8362 & 0.6931 & 0 & 0 & 2.3117 & 0.6931 & 5450 \\
        
        15\% & 6.8002 & 0.6931 & 0 & 0 & 3.6109 & 0.6931 & 5450 \\
        
        25\% & 7.1365 & 0.6931 & 0 & 0 & 4.4308 & 0.6931 & 5450 \\
        
        50\% & 7.6178 & 1.3863 & 1.6094 & 0.33 & 5.1985 & 0.6931 & 5450 \\
        
        75\% & 8.0034 & 1.6094 & 2.8332 & 1.25 & 5.8435 & 1.6094 & 5450 \\
        
        90\% & 8.5337 & 2.0794 & 3.7842 & 2.68 & 5.8999 & 2.7081 & 5450 \\
        
        95\% & 8.8785 & 2.8332 & 4.2767 & 3.125 & 5.9026 & 2.9839 & 5450 \\
        
        99\% & 9.3037 & 2.9839 & 5.0434 & 3.125 & 5.9026 & 2.9839 & 5450 \\
        
        max & 9.3037 & 2.9839 & 6.1137 & 3.125 & 5.9026 & 2.9839 & 5450 \\
        
        variance & 0.5387 & 0.3695 & 2.1644 & 0.9997 & 1.3582 & 0.5658 & 0 \\
        
        \bottomrule
    \end{tabular}
    \end{adjustbox}
    \end{scriptsize}
    \caption{ Estadísticos descriptivos de las variables numéricas tras el tratamiento de outliers con IQR (además de aplicar \lstinline[style=python]{np.expm1} a \texttt{price}). }
    \label{tab:df-iqr-desc-log}
\end{table}

Una versión sin la transformación logarítmica se incluye en el apéndice (Tabla \ref{tab:df-iqr-desc-no-log}).
