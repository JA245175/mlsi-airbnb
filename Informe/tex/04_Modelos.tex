\newpage

\section{Modelos de Regresión}

Todos los modelos se entrenaron mediante un \texttt{Pipeline} que combina el preprocesamiento (\texttt{StandardScaler} para numéricas y \texttt{OneHotEncoder} para categóricas) con el estimador correspondiente. La validación inicial se realizó con un hold-out 70/30 y semilla \texttt{random\_state=42}.

\lstinputlisting[language=Python, caption={Modelos de regresión.}]{./code/regresionModel.py}


\subsection{Modelos implementados}
Se implementaron y evaluaron los siguientes modelos de regresión:

\begin{itemize}
    \item \textbf{Dummy Regressor}: Utilizado como línea base (\textit{baseline}), prediciendo siempre la media del conjunto de entrenamiento.
    \item \textbf{Regresión Lineal}: Modelo base sin regularización.
    \item \textbf{Regresión Lasso (L1)}: Regresión lineal con regularización L1 para selección de características.
    \item \textbf{Regresión Ridge (L2)}: Regresión lineal con regularización L2 para manejar multicolinealidad.
    \item \textbf{Árbol de Decisión}: Modelo no lineal capaz de capturar relaciones complejas.
    \item \textbf{Random Forest}: Ensamble de árboles (Bagging) para reducir varianza y mejorar generalización.
    \item \textbf{Gradient Boosting}: Ensamble secuencial (Boosting) que optimiza los errores de los árboles previos.
    \item \textbf{Redes Neuronales (MLP)}: Perceptrón multicapa para capturar relaciones no lineales profundas.
\end{itemize}

\lstinputlisting[language=Python, caption={ Modelos de regresión con hiperparámetros utilizados.}]{./code/regresionModelHyperparam.py}

\subsection{Resultados iniciales en validación (70/30)}
Las métricas se calcularon sobre el precio transformado con \texttt{log1p} (Sección\nameref{sec:preprocesamiento}).


\begin{table}[H]
\centering
\begin{tabular}{lrrrr}
\toprule
Modelo & rmse & MSE & MAE & R2 \\
\midrule
Dummy & 0.6834 & 0.4670 & 0.5455 & -0.0001 \\
LinearRegression & 0.5441 & 0.2960 & 0.4205 & 0.3660 \\
DecisionTree & 0.7303 & 0.5334 & 0.5482 & -0.1422 \\
RandomForest & 0.5175 & 0.2678 & 0.3974 & 0.4265 \\
GradientBoosting & 0.5210 & 0.2715 & 0.4070 & 0.4186 \\
NeuralNetwork & 0.5390 & 0.2905 & 0.4203 & 0.3778 \\
\bottomrule
\end{tabular}

\caption{Resultados del primer barrido de modelos sobre la partición de validación (70/30).}
\label{tab:model-results-first-barrido}
\end{table}
