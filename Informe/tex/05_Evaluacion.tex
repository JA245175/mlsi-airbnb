\newpage

\section{Evaluación y selección de modelos}

La evaluación siguió dos etapas:

\begin{enumerate}
    \item Validación inicial con hold-out 70/30 para iterar rápidamente sobre modelos y preseleccionar candidatos.
    \item Validación cruzada con \texttt{GridSearchCV} (5 pliegues) para seleccionar hiperparámetros y evaluar la robustez de los modelos preseleccionados. 
\end{enumerate}

Además, todas las métricas se calcularon sobre el precio transformado con \texttt{log1p} para mantener consistencia con el entrenamiento.

\subsection{Estrategia y métricas}

Se utilizó \texttt{train\_test\_split} con \texttt{random\_state=42} y, para la búsqueda de hiperparámetros, \texttt{GridSearchCV} con \texttt{KFold} de 5 pliegues y \texttt{n\_jobs=-1}. La métrica principal fue RMSE; también se monitorizaron MSE, MAE y R\textsuperscript{2}.

\subsection{Hiperparámetros óptimos y desempeño en la partición de validación}

\subsubsection{Laso}


\begin{table}[H]
\centering
\begin{tabular}{rrrr}
\toprule
param\_model\_\_alpha & rmse\_mean & rmse\_std & MSE\_mean \\
\midrule
0.0010 & 0.5315 & 0.0102 & 0.2825 \\
0.0100 & 0.5351 & 0.0105 & 0.2864 \\
0.1000 & 0.5717 & 0.0094 & 0.3268 \\
1.0000 & 0.6829 & 0.0074 & 0.4664 \\
10.0000 & 0.6829 & 0.0074 & 0.4664 \\
100.0000 & 0.6829 & 0.0074 & 0.4664 \\
\bottomrule
\end{tabular}

\caption{Resultados del GridSearchCV para la regresión Lasso.}
\label{tab:lasso-gridsearch}
\end{table}


\subsubsection{Ridge}


\begin{table}[H]
\centering
\begin{tabular}{rrrr}
\toprule
param\_model\_\_alpha & rmse\_mean & rmse\_std & MSE\_mean \\
\midrule
0.1000 & 0.5884 & 0.0150 & 0.3462 \\
0.0100 & 0.5884 & 0.0150 & 0.3462 \\
0.0010 & 0.5884 & 0.0150 & 0.3462 \\
1.0000 & 0.5885 & 0.0149 & 0.3463 \\
10.0000 & 0.5888 & 0.0146 & 0.3467 \\
100.0000 & 0.5918 & 0.0146 & 0.3502 \\
\bottomrule
\end{tabular}

\caption{Resultados del GridSearchCV para regresión Ridge.}
\label{tab:ridge-gridsearch}
\end{table}


\subsubsection{Árbol de Decisión}


\begin{table}[H]
    \centering
    \setlength{\tabcolsep}{4pt}
        \begin{scriptsize}
\begin{adjustbox}{width=\textwidth}
    \begin{tabular}{lrrrrr}
\toprule
param\_model\_\_max\_depth & param\_model\_\_min\_samples\_split & param\_model\_\_min\_samples\_leaf & rmse\_mean & rmse\_std & MSE\_mean \\
\midrule
5 & 5 & 4 & 0.5490 & 0.0104 & 0.3014 \\
5 & 10 & 4 & 0.5490 & 0.0104 & 0.3014 \\
5 & 2 & 4 & 0.5490 & 0.0104 & 0.3014 \\
5 & 10 & 2 & 0.5491 & 0.0103 & 0.3015 \\
5 & 2 & 2 & 0.5491 & 0.0102 & 0.3015 \\
5 & 5 & 2 & 0.5491 & 0.0102 & 0.3015 \\
5 & 10 & 1 & 0.5492 & 0.0101 & 0.3016 \\
5 & 2 & 1 & 0.5493 & 0.0101 & 0.3017 \\
5 & 5 & 1 & 0.5493 & 0.0101 & 0.3017 \\
3 & 10 & 2 & 0.5620 & 0.0104 & 0.3159 \\
3 & 5 & 1 & 0.5620 & 0.0104 & 0.3159 \\
3 & 2 & 4 & 0.5620 & 0.0104 & 0.3159 \\
3 & 5 & 2 & 0.5620 & 0.0104 & 0.3159 \\
3 & 2 & 1 & 0.5620 & 0.0104 & 0.3159 \\
3 & 5 & 4 & 0.5620 & 0.0104 & 0.3159 \\
3 & 2 & 2 & 0.5620 & 0.0104 & 0.3159 \\
3 & 10 & 4 & 0.5620 & 0.0104 & 0.3159 \\
3 & 10 & 1 & 0.5620 & 0.0104 & 0.3159 \\
2 & 10 & 1 & 0.5725 & 0.0093 & 0.3278 \\
2 & 5 & 1 & 0.5725 & 0.0093 & 0.3278 \\
2 & 10 & 2 & 0.5725 & 0.0093 & 0.3278 \\
2 & 2 & 1 & 0.5725 & 0.0093 & 0.3278 \\
2 & 5 & 4 & 0.5725 & 0.0093 & 0.3278 \\
2 & 2 & 4 & 0.5725 & 0.0093 & 0.3278 \\
2 & 10 & 4 & 0.5725 & 0.0093 & 0.3278 \\
2 & 5 & 2 & 0.5725 & 0.0093 & 0.3278 \\
2 & 2 & 2 & 0.5725 & 0.0093 & 0.3278 \\
10 & 10 & 4 & 0.5764 & 0.0136 & 0.3322 \\
10 & 5 & 4 & 0.5773 & 0.0136 & 0.3332 \\
10 & 2 & 4 & 0.5773 & 0.0136 & 0.3332 \\
10 & 10 & 2 & 0.5777 & 0.0126 & 0.3338 \\
10 & 10 & 1 & 0.5814 & 0.0116 & 0.3380 \\
10 & 5 & 2 & 0.5829 & 0.0126 & 0.3398 \\
10 & 2 & 2 & 0.5831 & 0.0135 & 0.3400 \\
10 & 5 & 1 & 0.5878 & 0.0133 & 0.3455 \\
10 & 2 & 1 & 0.5885 & 0.0136 & 0.3464 \\
1 & 5 & 4 & 0.5903 & 0.0108 & 0.3485 \\
1 & 10 & 1 & 0.5903 & 0.0108 & 0.3485 \\
1 & 5 & 1 & 0.5903 & 0.0108 & 0.3485 \\
1 & 2 & 1 & 0.5903 & 0.0108 & 0.3485 \\
1 & 10 & 4 & 0.5903 & 0.0108 & 0.3485 \\
1 & 2 & 4 & 0.5903 & 0.0108 & 0.3485 \\
1 & 10 & 2 & 0.5903 & 0.0108 & 0.3485 \\
1 & 5 & 2 & 0.5903 & 0.0108 & 0.3485 \\
1 & 2 & 2 & 0.5903 & 0.0108 & 0.3485 \\
20 & 10 & 4 & 0.6459 & 0.0099 & 0.4172 \\
NaN & 10 & 4 & 0.6517 & 0.0089 & 0.4247 \\
20 & 5 & 4 & 0.6532 & 0.0087 & 0.4266 \\
20 & 2 & 4 & 0.6532 & 0.0087 & 0.4266 \\
20 & 10 & 2 & 0.6582 & 0.0142 & 0.4333 \\
NaN & 5 & 4 & 0.6613 & 0.0078 & 0.4374 \\
NaN & 2 & 4 & 0.6613 & 0.0078 & 0.4374 \\
20 & 10 & 1 & 0.6653 & 0.0099 & 0.4427 \\
NaN & 10 & 2 & 0.6666 & 0.0128 & 0.4443 \\
NaN & 10 & 1 & 0.6775 & 0.0067 & 0.4591 \\
20 & 5 & 2 & 0.6877 & 0.0114 & 0.4730 \\
20 & 5 & 1 & 0.6949 & 0.0154 & 0.4828 \\
20 & 2 & 2 & 0.6953 & 0.0157 & 0.4835 \\
NaN & 5 & 2 & 0.7009 & 0.0106 & 0.4913 \\
NaN & 2 & 2 & 0.7085 & 0.0093 & 0.5020 \\
20 & 2 & 1 & 0.7128 & 0.0158 & 0.5080 \\
NaN & 5 & 1 & 0.7170 & 0.0104 & 0.5142 \\
NaN & 2 & 1 & 0.7366 & 0.0150 & 0.5426 \\
\bottomrule
\end{tabular}

    \end{adjustbox}
    \end{scriptsize}
    \caption{Resultados del GridSearchCV para el modelo Decision Tree Regressor.}
    \label{tab:decisiontreeregressor-gridsearch}
\end{table}


\subsubsection{Random Forest}


\begin{table}[H]
    \centering
    \setlength{\tabcolsep}{4pt}
        \begin{scriptsize}
\begin{adjustbox}{width=\textwidth}
    \begin{tabular}{rlrrrrr}
\toprule
param\_model\_\_n\_estimators & param\_model\_\_max\_depth & param\_model\_\_min\_samples\_split & param\_model\_\_min\_samples\_leaf & rmse\_mean & rmse\_std & MSE\_mean \\
\midrule
200 & 10 & 5 & 1 & 0.5223 & 0.0112 & 0.2728 \\
200 & 10 & 2 & 1 & 0.5224 & 0.0112 & 0.2729 \\
200 & 20 & 5 & 2 & 0.5224 & 0.0093 & 0.2729 \\
200 & 10 & 5 & 2 & 0.5224 & 0.0111 & 0.2729 \\
200 & 10 & 2 & 2 & 0.5225 & 0.0111 & 0.2730 \\
200 & 20 & 5 & 1 & 0.5226 & 0.0089 & 0.2731 \\
200 & 20 & 2 & 2 & 0.5226 & 0.0092 & 0.2731 \\
100 & 10 & 5 & 1 & 0.5230 & 0.0117 & 0.2735 \\
200 & NaN & 5 & 2 & 0.5230 & 0.0092 & 0.2735 \\
200 & 20 & 2 & 1 & 0.5231 & 0.0087 & 0.2736 \\
100 & 10 & 5 & 2 & 0.5231 & 0.0114 & 0.2737 \\
200 & NaN & 2 & 2 & 0.5232 & 0.0091 & 0.2737 \\
100 & 10 & 2 & 1 & 0.5232 & 0.0115 & 0.2737 \\
100 & 10 & 2 & 2 & 0.5232 & 0.0113 & 0.2738 \\
200 & NaN & 5 & 1 & 0.5233 & 0.0086 & 0.2738 \\
50 & 10 & 5 & 1 & 0.5238 & 0.0119 & 0.2743 \\
50 & 10 & 5 & 2 & 0.5240 & 0.0116 & 0.2745 \\
50 & 10 & 2 & 1 & 0.5240 & 0.0115 & 0.2746 \\
50 & 10 & 2 & 2 & 0.5240 & 0.0115 & 0.2746 \\
100 & 20 & 5 & 1 & 0.5240 & 0.0092 & 0.2746 \\
100 & 20 & 5 & 2 & 0.5241 & 0.0094 & 0.2747 \\
200 & NaN & 2 & 1 & 0.5242 & 0.0086 & 0.2748 \\
100 & 20 & 2 & 1 & 0.5243 & 0.0094 & 0.2749 \\
100 & 20 & 2 & 2 & 0.5244 & 0.0092 & 0.2750 \\
100 & NaN & 5 & 2 & 0.5246 & 0.0094 & 0.2752 \\
100 & NaN & 5 & 1 & 0.5247 & 0.0091 & 0.2753 \\
100 & NaN & 2 & 2 & 0.5248 & 0.0094 & 0.2754 \\
100 & NaN & 2 & 1 & 0.5257 & 0.0090 & 0.2764 \\
50 & 20 & 5 & 1 & 0.5263 & 0.0095 & 0.2770 \\
50 & 20 & 5 & 2 & 0.5264 & 0.0095 & 0.2770 \\
50 & 20 & 2 & 2 & 0.5266 & 0.0094 & 0.2773 \\
50 & 20 & 2 & 1 & 0.5267 & 0.0099 & 0.2774 \\
50 & NaN & 5 & 1 & 0.5271 & 0.0096 & 0.2779 \\
50 & NaN & 5 & 2 & 0.5272 & 0.0095 & 0.2779 \\
50 & NaN & 2 & 2 & 0.5273 & 0.0095 & 0.2780 \\
50 & NaN & 2 & 1 & 0.5284 & 0.0094 & 0.2792 \\
200 & 5 & 2 & 2 & 0.5376 & 0.0119 & 0.2891 \\
200 & 5 & 5 & 2 & 0.5376 & 0.0119 & 0.2891 \\
200 & 5 & 2 & 1 & 0.5377 & 0.0119 & 0.2891 \\
200 & 5 & 5 & 1 & 0.5377 & 0.0119 & 0.2892 \\
100 & 5 & 2 & 2 & 0.5379 & 0.0118 & 0.2893 \\
100 & 5 & 5 & 2 & 0.5379 & 0.0118 & 0.2893 \\
100 & 5 & 2 & 1 & 0.5379 & 0.0118 & 0.2894 \\
100 & 5 & 5 & 1 & 0.5379 & 0.0118 & 0.2894 \\
50 & 5 & 2 & 2 & 0.5382 & 0.0119 & 0.2897 \\
50 & 5 & 5 & 2 & 0.5382 & 0.0119 & 0.2897 \\
50 & 5 & 2 & 1 & 0.5383 & 0.0119 & 0.2897 \\
50 & 5 & 5 & 1 & 0.5383 & 0.0120 & 0.2897 \\
1 & 5 & 2 & 2 & 0.5526 & 0.0130 & 0.3054 \\
1 & 5 & 5 & 2 & 0.5526 & 0.0130 & 0.3054 \\
1 & 5 & 2 & 1 & 0.5534 & 0.0130 & 0.3063 \\
1 & 5 & 5 & 1 & 0.5534 & 0.0130 & 0.3063 \\
200 & 3 & 2 & 2 & 0.5550 & 0.0116 & 0.3080 \\
200 & 3 & 5 & 2 & 0.5550 & 0.0116 & 0.3080 \\
200 & 3 & 5 & 1 & 0.5550 & 0.0116 & 0.3081 \\
200 & 3 & 2 & 1 & 0.5550 & 0.0116 & 0.3081 \\
50 & 3 & 2 & 2 & 0.5551 & 0.0117 & 0.3081 \\
50 & 3 & 5 & 2 & 0.5551 & 0.0117 & 0.3081 \\
50 & 3 & 5 & 1 & 0.5551 & 0.0117 & 0.3081 \\
50 & 3 & 2 & 1 & 0.5551 & 0.0117 & 0.3081 \\
100 & 3 & 5 & 2 & 0.5551 & 0.0115 & 0.3081 \\
100 & 3 & 2 & 2 & 0.5551 & 0.0115 & 0.3081 \\
100 & 3 & 2 & 1 & 0.5551 & 0.0115 & 0.3081 \\
100 & 3 & 5 & 1 & 0.5551 & 0.0115 & 0.3081 \\
1 & 3 & 5 & 2 & 0.5642 & 0.0146 & 0.3184 \\
1 & 3 & 2 & 2 & 0.5642 & 0.0146 & 0.3184 \\
1 & 3 & 2 & 1 & 0.5642 & 0.0146 & 0.3184 \\
1 & 3 & 5 & 1 & 0.5642 & 0.0146 & 0.3184 \\
50 & 1 & 2 & 2 & 0.5881 & 0.0108 & 0.3459 \\
50 & 1 & 5 & 2 & 0.5881 & 0.0108 & 0.3459 \\
50 & 1 & 5 & 1 & 0.5881 & 0.0108 & 0.3459 \\
50 & 1 & 2 & 1 & 0.5881 & 0.0108 & 0.3459 \\
200 & 1 & 2 & 1 & 0.5882 & 0.0110 & 0.3459 \\
200 & 1 & 5 & 1 & 0.5882 & 0.0110 & 0.3459 \\
200 & 1 & 5 & 2 & 0.5882 & 0.0110 & 0.3459 \\
200 & 1 & 2 & 2 & 0.5882 & 0.0110 & 0.3459 \\
100 & 1 & 5 & 1 & 0.5882 & 0.0109 & 0.3459 \\
100 & 1 & 2 & 1 & 0.5882 & 0.0109 & 0.3459 \\
100 & 1 & 2 & 2 & 0.5882 & 0.0109 & 0.3459 \\
100 & 1 & 5 & 2 & 0.5882 & 0.0109 & 0.3459 \\
1 & 1 & 5 & 2 & 0.5895 & 0.0108 & 0.3475 \\
1 & 1 & 2 & 2 & 0.5895 & 0.0108 & 0.3475 \\
1 & 1 & 2 & 1 & 0.5895 & 0.0108 & 0.3475 \\
1 & 1 & 5 & 1 & 0.5895 & 0.0108 & 0.3475 \\
1 & 10 & 5 & 2 & 0.5981 & 0.0147 & 0.3577 \\
1 & 10 & 2 & 2 & 0.6011 & 0.0139 & 0.3613 \\
1 & 10 & 5 & 1 & 0.6028 & 0.0221 & 0.3634 \\
1 & 10 & 2 & 1 & 0.6037 & 0.0213 & 0.3644 \\
1 & 20 & 5 & 2 & 0.7011 & 0.0143 & 0.4916 \\
1 & NaN & 5 & 2 & 0.7081 & 0.0144 & 0.5014 \\
1 & 20 & 2 & 2 & 0.7084 & 0.0122 & 0.5018 \\
1 & 20 & 5 & 1 & 0.7114 & 0.0164 & 0.5060 \\
1 & NaN & 2 & 2 & 0.7160 & 0.0112 & 0.5127 \\
1 & NaN & 5 & 1 & 0.7237 & 0.0117 & 0.5238 \\
1 & 20 & 2 & 1 & 0.7318 & 0.0205 & 0.5356 \\
1 & NaN & 2 & 1 & 0.7453 & 0.0142 & 0.5554 \\
\bottomrule
\end{tabular}

    \end{adjustbox}
    \end{scriptsize}
    \caption{Resultados del GridSearchCV para el modelo Random Forest Regressor.}
    \label{tab:randomforestregressor-gridsearch}
\end{table}


\subsubsection{Gradient Boosting}


\begin{table}[H]
    \centering
    \setlength{\tabcolsep}{4pt}
        \begin{scriptsize}
\begin{adjustbox}{width=\textwidth}
    \begin{tabular}{rlrrrrr}
\toprule
param\_model\_\_n\_estimators & param\_model\_\_max\_depth & param\_model\_\_min\_samples\_split & param\_model\_\_min\_samples\_leaf & rmse\_mean & rmse\_std & MSE\_mean \\
\midrule
200 & 10 & 5 & 2 & 0.5243 & 0.0111 & 0.2748 \\
200 & 10 & 2 & 2 & 0.5243 & 0.0111 & 0.2749 \\
200 & 10 & 2 & 1 & 0.5243 & 0.0113 & 0.2749 \\
200 & 10 & 5 & 1 & 0.5243 & 0.0113 & 0.2749 \\
100 & 10 & 2 & 1 & 0.5245 & 0.0116 & 0.2751 \\
100 & 10 & 5 & 1 & 0.5245 & 0.0116 & 0.2751 \\
100 & 10 & 2 & 2 & 0.5245 & 0.0112 & 0.2751 \\
100 & 10 & 5 & 2 & 0.5246 & 0.0112 & 0.2752 \\
50 & 10 & 5 & 2 & 0.5251 & 0.0112 & 0.2758 \\
50 & 10 & 2 & 2 & 0.5252 & 0.0111 & 0.2758 \\
50 & 10 & 2 & 1 & 0.5253 & 0.0113 & 0.2759 \\
50 & 10 & 5 & 1 & 0.5253 & 0.0114 & 0.2760 \\
200 & 20 & 5 & 2 & 0.5284 & 0.0101 & 0.2792 \\
200 & 20 & 5 & 1 & 0.5284 & 0.0103 & 0.2792 \\
200 & 20 & 2 & 2 & 0.5286 & 0.0101 & 0.2794 \\
200 & NaN & 5 & 2 & 0.5287 & 0.0099 & 0.2796 \\
200 & NaN & 5 & 1 & 0.5290 & 0.0103 & 0.2799 \\
200 & NaN & 2 & 2 & 0.5292 & 0.0100 & 0.2801 \\
200 & 20 & 2 & 1 & 0.5292 & 0.0102 & 0.2801 \\
100 & 20 & 5 & 2 & 0.5293 & 0.0104 & 0.2802 \\
100 & 20 & 5 & 1 & 0.5296 & 0.0105 & 0.2805 \\
100 & 20 & 2 & 2 & 0.5298 & 0.0103 & 0.2806 \\
100 & NaN & 5 & 2 & 0.5299 & 0.0102 & 0.2808 \\
200 & NaN & 2 & 1 & 0.5301 & 0.0104 & 0.2810 \\
100 & 20 & 2 & 1 & 0.5302 & 0.0104 & 0.2811 \\
100 & NaN & 5 & 1 & 0.5304 & 0.0105 & 0.2813 \\
100 & NaN & 2 & 2 & 0.5306 & 0.0103 & 0.2815 \\
100 & NaN & 2 & 1 & 0.5315 & 0.0104 & 0.2825 \\
50 & 20 & 5 & 2 & 0.5322 & 0.0100 & 0.2832 \\
50 & 20 & 2 & 2 & 0.5327 & 0.0101 & 0.2838 \\
50 & NaN & 5 & 2 & 0.5329 & 0.0097 & 0.2840 \\
50 & 20 & 5 & 1 & 0.5331 & 0.0103 & 0.2842 \\
50 & NaN & 5 & 1 & 0.5337 & 0.0099 & 0.2848 \\
50 & NaN & 2 & 2 & 0.5338 & 0.0101 & 0.2850 \\
50 & 20 & 2 & 1 & 0.5340 & 0.0104 & 0.2851 \\
50 & NaN & 2 & 1 & 0.5349 & 0.0104 & 0.2861 \\
200 & 5 & 2 & 2 & 0.5403 & 0.0124 & 0.2919 \\
200 & 5 & 5 & 2 & 0.5403 & 0.0124 & 0.2919 \\
200 & 5 & 2 & 1 & 0.5403 & 0.0125 & 0.2919 \\
200 & 5 & 5 & 1 & 0.5403 & 0.0124 & 0.2920 \\
50 & 5 & 5 & 2 & 0.5405 & 0.0124 & 0.2921 \\
50 & 5 & 2 & 2 & 0.5405 & 0.0124 & 0.2921 \\
100 & 5 & 2 & 2 & 0.5406 & 0.0124 & 0.2922 \\
100 & 5 & 5 & 2 & 0.5406 & 0.0124 & 0.2922 \\
50 & 5 & 2 & 1 & 0.5406 & 0.0125 & 0.2922 \\
50 & 5 & 5 & 1 & 0.5406 & 0.0125 & 0.2922 \\
100 & 5 & 2 & 1 & 0.5406 & 0.0125 & 0.2922 \\
100 & 5 & 5 & 1 & 0.5406 & 0.0124 & 0.2923 \\
1 & 5 & 5 & 2 & 0.5566 & 0.0130 & 0.3097 \\
1 & 5 & 2 & 2 & 0.5567 & 0.0132 & 0.3099 \\
1 & 5 & 5 & 1 & 0.5569 & 0.0140 & 0.3101 \\
1 & 5 & 2 & 1 & 0.5570 & 0.0142 & 0.3103 \\
200 & 3 & 2 & 2 & 0.5618 & 0.0126 & 0.3156 \\
200 & 3 & 5 & 2 & 0.5618 & 0.0126 & 0.3156 \\
200 & 3 & 5 & 1 & 0.5618 & 0.0126 & 0.3156 \\
200 & 3 & 2 & 1 & 0.5618 & 0.0126 & 0.3156 \\
50 & 3 & 2 & 2 & 0.5619 & 0.0126 & 0.3157 \\
50 & 3 & 5 & 2 & 0.5619 & 0.0126 & 0.3157 \\
50 & 3 & 5 & 1 & 0.5619 & 0.0126 & 0.3158 \\
50 & 3 & 2 & 1 & 0.5619 & 0.0126 & 0.3158 \\
100 & 3 & 5 & 2 & 0.5620 & 0.0127 & 0.3158 \\
100 & 3 & 2 & 2 & 0.5620 & 0.0127 & 0.3158 \\
100 & 3 & 2 & 1 & 0.5620 & 0.0127 & 0.3158 \\
100 & 3 & 5 & 1 & 0.5620 & 0.0127 & 0.3158 \\
1 & 3 & 5 & 2 & 0.5695 & 0.0120 & 0.3243 \\
1 & 3 & 2 & 2 & 0.5695 & 0.0120 & 0.3243 \\
1 & 3 & 2 & 1 & 0.5695 & 0.0120 & 0.3243 \\
1 & 3 & 5 & 1 & 0.5695 & 0.0120 & 0.3243 \\
1 & 10 & 5 & 2 & 0.5914 & 0.0100 & 0.3498 \\
1 & 10 & 2 & 2 & 0.5921 & 0.0106 & 0.3506 \\
1 & 10 & 5 & 1 & 0.5955 & 0.0125 & 0.3546 \\
1 & 10 & 2 & 1 & 0.5970 & 0.0113 & 0.3565 \\
100 & 1 & 2 & 2 & 0.5997 & 0.0138 & 0.3597 \\
100 & 1 & 5 & 2 & 0.5997 & 0.0138 & 0.3597 \\
100 & 1 & 5 & 1 & 0.5997 & 0.0138 & 0.3597 \\
100 & 1 & 2 & 1 & 0.5997 & 0.0138 & 0.3597 \\
200 & 1 & 2 & 1 & 0.5997 & 0.0138 & 0.3597 \\
200 & 1 & 5 & 1 & 0.5997 & 0.0138 & 0.3597 \\
200 & 1 & 5 & 2 & 0.5997 & 0.0138 & 0.3597 \\
200 & 1 & 2 & 2 & 0.5997 & 0.0138 & 0.3597 \\
50 & 1 & 5 & 1 & 0.5997 & 0.0137 & 0.3597 \\
50 & 1 & 2 & 1 & 0.5997 & 0.0137 & 0.3597 \\
50 & 1 & 2 & 2 & 0.5997 & 0.0137 & 0.3597 \\
50 & 1 & 5 & 2 & 0.5997 & 0.0137 & 0.3597 \\
1 & 1 & 5 & 2 & 0.5998 & 0.0138 & 0.3597 \\
1 & 1 & 2 & 2 & 0.5998 & 0.0138 & 0.3597 \\
1 & 1 & 2 & 1 & 0.5998 & 0.0138 & 0.3597 \\
1 & 1 & 5 & 1 & 0.5998 & 0.0138 & 0.3597 \\
1 & 20 & 5 & 2 & 0.6985 & 0.0123 & 0.4880 \\
1 & NaN & 5 & 2 & 0.7048 & 0.0104 & 0.4967 \\
1 & 20 & 2 & 2 & 0.7058 & 0.0113 & 0.4982 \\
1 & NaN & 2 & 2 & 0.7132 & 0.0123 & 0.5087 \\
1 & 20 & 5 & 1 & 0.7165 & 0.0136 & 0.5133 \\
1 & NaN & 5 & 1 & 0.7269 & 0.0117 & 0.5284 \\
1 & 20 & 2 & 1 & 0.7393 & 0.0134 & 0.5466 \\
1 & NaN & 2 & 1 & 0.7521 & 0.0125 & 0.5656 \\
\bottomrule
\end{tabular}

    \end{adjustbox}
    \end{scriptsize}
    \caption{Resultados del GridSearchCV para el modelo Gradient Boosting.}
    \label{tab:GradientBoostingRegressor-gridsearch}
\end{table}



\subsubsection{Red Neuronal (MLP)}

\input{./tex/tables/NeuralNetwork_gridsearch_results.tex}


\subsection{Resultados Finales}

A continuación se presentan los mejores hiperparámetros encontrados para cada modelo junto con sus métricas en la partición de validación 70/30:


\begin{table}[H]
    \centering
    \setlength{\tabcolsep}{4pt}
        \begin{scriptsize}
\begin{adjustbox}{width=\textwidth}
    \begin{tabular}{lrrrrr}
\toprule
Modelo & rmse\_mean & rmse\_std & MSE\_mean & MAE\_mean & R2\_mean \\
\midrule
GradientBoosting & 0.5167 & 0.0063 & 0.2670 & 0.4026 & 0.4274 \\
RandomForest & 0.5171 & 0.0075 & 0.2675 & 0.4014 & 0.4264 \\
NeuralNetwork & 0.5342 & 0.0059 & 0.2854 & 0.4168 & 0.3879 \\
LinearRegression\_Ridge & 0.5416 & 0.0058 & 0.2933 & 0.4218 & 0.3709 \\
LinearRegression & 0.5416 & 0.0058 & 0.2934 & 0.4218 & 0.3708 \\
LinearRegression\_Lasso & 0.5425 & 0.0057 & 0.2943 & 0.4234 & 0.3688 \\
DecisionTree & 0.5482 & 0.0070 & 0.3006 & 0.4280 & 0.3554 \\
Dummy & 0.6831 & 0.0073 & 0.4667 & 0.5468 & -0.0008 \\
\bottomrule
\end{tabular}

    \end{adjustbox}
    \end{scriptsize}
    \caption{Resultados de validación cruzada (5-fold) para todos los modelos.}
    \label{tab:cv-all-models}
\end{table}


\subsubsection{Síntesis de resultados}
\begin{itemize}
    \item \textbf{Hold-out 70/30} (Tabla \ref{tab:model-results-first-barrido}): Random Forest (RMSE 0.5175, MAE 0.3974, R\textsuperscript{2} 0.4265) y Gradient Boosting (0.5210, 0.4070, 0.4186) lideran; la red neuronal queda cerca (RMSE 0.5390). Lasso y Ridge rondan RMSE 0.54 con R\textsuperscript{2} $\approx$ 0.37; el árbol simple se degrada (RMSE 0.73, R\textsuperscript{2} negativo).
    \item \textbf{Validación cruzada (5 pliegues)} (Tabla \ref{tab:cv-all-models}): el ranking se mantiene con baja varianza; Gradient Boosting logra RMSE \(0.5167 \pm 0.0063\), Random Forest \(0.5171 \pm 0.0075\); el MLP queda en $\sim$0.534 y Lasso/Ridge en $\sim$0.542; el Dummy queda muy atrás.
    \item \textbf{Benchmark de hiperparámetros}: Lasso y Ridge prefieren \texttt{alpha=0.001} (Tablas \ref{tab:lasso-gridsearch} y \ref{tab:ridge-gridsearch}); el Árbol de Decisión rinde mejor con \texttt{max\_depth=5} y \texttt{min\_samples\_leaf=4} (Tabla \ref{tab:decisiontreeregressor-gridsearch}); Random Forest mejora con \texttt{n\_estimators=200}, \texttt{max\_depth=10} y hojas pequeñas (Tabla \ref{tab:randomforestregressor-gridsearch}). Las ganancias entre configuraciones fuertes y débiles son moderadas (~0.02--0.03 RMSE) pero consistentes.
    \item \textbf{Conclusión práctica}: Random Forest y Gradient Boosting son los candidatos finales; los lineales quedan como baselines fuertes y el MLP es competitivo pero ligeramente detrás. Las predicciones finales se revirtieron con \texttt{np.expm1} antes de generar los archivos de envío en \texttt{pred/}.
\end{itemize}
