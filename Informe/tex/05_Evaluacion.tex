\newpage

\section{Evaluación y selección de modelos}

La evaluación siguió dos etapas:

\begin{enumerate}
    \item Validación inicial con hold-out 70/30 para iterar rápidamente sobre modelos y preseleccionar candidatos.
    \item Validación cruzada con \texttt{GridSearchCV} (5 pliegues) para seleccionar hiperparámetros y evaluar la robustez de los modelos preseleccionados. 
\end{enumerate}

Además, todas las métricas se calcularon sobre el precio transformado con \texttt{log1p} para mantener consistencia con el entrenamiento.

\subsection{Estrategia y métricas}

Se utilizó \texttt{train\_test\_split} con \texttt{random\_state=42} y, para la búsqueda de hiperparámetros, \texttt{GridSearchCV} con \texttt{KFold} de 5 pliegues y \texttt{n\_jobs=-1}. La métrica principal fue RMSE; también se monitorizaron MSE, MAE y R\textsuperscript{2}.

\subsection{Hiperparámetros óptimos y desempeño en la partición de validación}

\subsubsection{Laso}


\begin{table}[H]
\centering
\begin{tabular}{rrrr}
\toprule
param\_model\_\_alpha & rmse\_mean & rmse\_std & MSE\_mean \\
\midrule
0.0010 & 0.5315 & 0.0102 & 0.2825 \\
0.0100 & 0.5351 & 0.0105 & 0.2864 \\
0.1000 & 0.5717 & 0.0094 & 0.3268 \\
1.0000 & 0.6829 & 0.0074 & 0.4664 \\
10.0000 & 0.6829 & 0.0074 & 0.4664 \\
100.0000 & 0.6829 & 0.0074 & 0.4664 \\
\bottomrule
\end{tabular}

\caption{Resultados del GridSearchCV para la regresión Lasso.}
\label{tab:lasso-gridsearch}
\end{table}


\subsubsection{Ridge}


\begin{table}[H]
\centering
\begin{tabular}{rrrr}
\toprule
param\_model\_\_alpha & rmse\_mean & rmse\_std & MSE\_mean \\
\midrule
0.1000 & 0.5884 & 0.0150 & 0.3462 \\
0.0100 & 0.5884 & 0.0150 & 0.3462 \\
0.0010 & 0.5884 & 0.0150 & 0.3462 \\
1.0000 & 0.5885 & 0.0149 & 0.3463 \\
10.0000 & 0.5888 & 0.0146 & 0.3467 \\
100.0000 & 0.5918 & 0.0146 & 0.3502 \\
\bottomrule
\end{tabular}

\caption{Resultados del GridSearchCV para regresión Ridge.}
\label{tab:ridge-gridsearch}
\end{table}


\subsubsection{Árbol de Decisión}


\begin{table}[H]
    \centering
    \setlength{\tabcolsep}{4pt}
        \begin{scriptsize}
\begin{adjustbox}{width=\textwidth}
    \begin{tabular}{lrrrrr}
\toprule
param\_model\_\_max\_depth & param\_model\_\_min\_samples\_split & param\_model\_\_min\_samples\_leaf & rmse\_mean & rmse\_std & MSE\_mean \\
\midrule
5 & 5 & 4 & 0.5490 & 0.0104 & 0.3014 \\
5 & 10 & 4 & 0.5490 & 0.0104 & 0.3014 \\
5 & 2 & 4 & 0.5490 & 0.0104 & 0.3014 \\
5 & 10 & 2 & 0.5491 & 0.0103 & 0.3015 \\
5 & 2 & 2 & 0.5491 & 0.0102 & 0.3015 \\
5 & 5 & 2 & 0.5491 & 0.0102 & 0.3015 \\
5 & 10 & 1 & 0.5492 & 0.0101 & 0.3016 \\
5 & 2 & 1 & 0.5493 & 0.0101 & 0.3017 \\
5 & 5 & 1 & 0.5493 & 0.0101 & 0.3017 \\
3 & 10 & 2 & 0.5620 & 0.0104 & 0.3159 \\
3 & 5 & 1 & 0.5620 & 0.0104 & 0.3159 \\
3 & 2 & 4 & 0.5620 & 0.0104 & 0.3159 \\
3 & 5 & 2 & 0.5620 & 0.0104 & 0.3159 \\
3 & 2 & 1 & 0.5620 & 0.0104 & 0.3159 \\
3 & 5 & 4 & 0.5620 & 0.0104 & 0.3159 \\
3 & 2 & 2 & 0.5620 & 0.0104 & 0.3159 \\
3 & 10 & 4 & 0.5620 & 0.0104 & 0.3159 \\
3 & 10 & 1 & 0.5620 & 0.0104 & 0.3159 \\
2 & 10 & 1 & 0.5725 & 0.0093 & 0.3278 \\
2 & 5 & 1 & 0.5725 & 0.0093 & 0.3278 \\
2 & 10 & 2 & 0.5725 & 0.0093 & 0.3278 \\
2 & 2 & 1 & 0.5725 & 0.0093 & 0.3278 \\
2 & 5 & 4 & 0.5725 & 0.0093 & 0.3278 \\
2 & 2 & 4 & 0.5725 & 0.0093 & 0.3278 \\
2 & 10 & 4 & 0.5725 & 0.0093 & 0.3278 \\
2 & 5 & 2 & 0.5725 & 0.0093 & 0.3278 \\
2 & 2 & 2 & 0.5725 & 0.0093 & 0.3278 \\
10 & 10 & 4 & 0.5764 & 0.0136 & 0.3322 \\
10 & 5 & 4 & 0.5773 & 0.0136 & 0.3332 \\
10 & 2 & 4 & 0.5773 & 0.0136 & 0.3332 \\
10 & 10 & 2 & 0.5777 & 0.0126 & 0.3338 \\
10 & 10 & 1 & 0.5814 & 0.0116 & 0.3380 \\
10 & 5 & 2 & 0.5829 & 0.0126 & 0.3398 \\
10 & 2 & 2 & 0.5831 & 0.0135 & 0.3400 \\
10 & 5 & 1 & 0.5878 & 0.0133 & 0.3455 \\
10 & 2 & 1 & 0.5885 & 0.0136 & 0.3464 \\
1 & 5 & 4 & 0.5903 & 0.0108 & 0.3485 \\
1 & 10 & 1 & 0.5903 & 0.0108 & 0.3485 \\
1 & 5 & 1 & 0.5903 & 0.0108 & 0.3485 \\
1 & 2 & 1 & 0.5903 & 0.0108 & 0.3485 \\
1 & 10 & 4 & 0.5903 & 0.0108 & 0.3485 \\
1 & 2 & 4 & 0.5903 & 0.0108 & 0.3485 \\
1 & 10 & 2 & 0.5903 & 0.0108 & 0.3485 \\
1 & 5 & 2 & 0.5903 & 0.0108 & 0.3485 \\
1 & 2 & 2 & 0.5903 & 0.0108 & 0.3485 \\
20 & 10 & 4 & 0.6459 & 0.0099 & 0.4172 \\
NaN & 10 & 4 & 0.6517 & 0.0089 & 0.4247 \\
20 & 5 & 4 & 0.6532 & 0.0087 & 0.4266 \\
20 & 2 & 4 & 0.6532 & 0.0087 & 0.4266 \\
20 & 10 & 2 & 0.6582 & 0.0142 & 0.4333 \\
NaN & 5 & 4 & 0.6613 & 0.0078 & 0.4374 \\
NaN & 2 & 4 & 0.6613 & 0.0078 & 0.4374 \\
20 & 10 & 1 & 0.6653 & 0.0099 & 0.4427 \\
NaN & 10 & 2 & 0.6666 & 0.0128 & 0.4443 \\
NaN & 10 & 1 & 0.6775 & 0.0067 & 0.4591 \\
20 & 5 & 2 & 0.6877 & 0.0114 & 0.4730 \\
20 & 5 & 1 & 0.6949 & 0.0154 & 0.4828 \\
20 & 2 & 2 & 0.6953 & 0.0157 & 0.4835 \\
NaN & 5 & 2 & 0.7009 & 0.0106 & 0.4913 \\
NaN & 2 & 2 & 0.7085 & 0.0093 & 0.5020 \\
20 & 2 & 1 & 0.7128 & 0.0158 & 0.5080 \\
NaN & 5 & 1 & 0.7170 & 0.0104 & 0.5142 \\
NaN & 2 & 1 & 0.7366 & 0.0150 & 0.5426 \\
\bottomrule
\end{tabular}

    \end{adjustbox}
    \end{scriptsize}
    \caption{Resultados del GridSearchCV para el modelo Decision Tree Regressor.}
    \label{tab:decisiontreeregressor-gridsearch}
\end{table}


\subsubsection{Random Forest}


\begin{table}[H]
\centering
\setlength{\tabcolsep}{4pt}
\begin{scriptsize}
\begin{adjustbox}{width=\textwidth}
\begin{tabular}{lrrrrrr}
\toprule
param\_model\_\_max\_depth & param\_model\_\_min\_samples\_leaf & param\_model\_\_min\_samples\_split & param\_model\_\_n\_estimators & rmse\_mean & rmse\_std & MSE\_mean \\
\midrule
20 & 2 & 5 & 200 & 0.5703 & 0.0165 & 0.3252 \\
20 & 2 & 2 & 200 & 0.5704 & 0.0166 & 0.3254 \\
NaN & 2 & 2 & 200 & 0.5705 & 0.0164 & 0.3255 \\
NaN & 2 & 5 & 200 & 0.5705 & 0.0163 & 0.3255 \\
20 & 1 & 5 & 200 & 0.5707 & 0.0164 & 0.3257 \\
20 & 1 & 2 & 200 & 0.5710 & 0.0162 & 0.3260 \\
NaN & 1 & 5 & 200 & 0.5712 & 0.0161 & 0.3262 \\
NaN & 1 & 2 & 200 & 0.5713 & 0.0158 & 0.3264 \\
20 & 2 & 2 & 100 & 0.5718 & 0.0165 & 0.3270 \\
20 & 2 & 5 & 100 & 0.5719 & 0.0164 & 0.3271 \\
NaN & 2 & 5 & 100 & 0.5721 & 0.0161 & 0.3273 \\
NaN & 2 & 2 & 100 & 0.5721 & 0.0162 & 0.3273 \\
10 & 2 & 2 & 200 & 0.5722 & 0.0177 & 0.3274 \\
10 & 2 & 5 & 200 & 0.5722 & 0.0177 & 0.3274 \\
10 & 1 & 2 & 200 & 0.5723 & 0.0176 & 0.3276 \\
20 & 1 & 5 & 100 & 0.5724 & 0.0157 & 0.3277 \\
10 & 1 & 5 & 200 & 0.5725 & 0.0176 & 0.3277 \\
20 & 1 & 2 & 100 & 0.5728 & 0.0156 & 0.3281 \\
10 & 2 & 2 & 100 & 0.5729 & 0.0173 & 0.3282 \\
NaN & 1 & 5 & 100 & 0.5730 & 0.0153 & 0.3283 \\
10 & 2 & 5 & 100 & 0.5730 & 0.0172 & 0.3284 \\
10 & 1 & 2 & 100 & 0.5731 & 0.0170 & 0.3285 \\
10 & 1 & 5 & 100 & 0.5732 & 0.0172 & 0.3286 \\
NaN & 1 & 2 & 100 & 0.5733 & 0.0150 & 0.3287 \\
10 & 2 & 2 & 50 & 0.5734 & 0.0177 & 0.3288 \\
10 & 2 & 5 & 50 & 0.5736 & 0.0176 & 0.3291 \\
10 & 1 & 5 & 50 & 0.5738 & 0.0177 & 0.3292 \\
20 & 2 & 5 & 50 & 0.5738 & 0.0176 & 0.3293 \\
20 & 2 & 2 & 50 & 0.5738 & 0.0180 & 0.3293 \\
NaN & 2 & 5 & 50 & 0.5740 & 0.0174 & 0.3295 \\
10 & 1 & 2 & 50 & 0.5740 & 0.0171 & 0.3295 \\
NaN & 2 & 2 & 50 & 0.5743 & 0.0173 & 0.3298 \\
20 & 1 & 5 & 50 & 0.5745 & 0.0164 & 0.3301 \\
NaN & 1 & 5 & 50 & 0.5747 & 0.0159 & 0.3302 \\
20 & 1 & 2 & 50 & 0.5748 & 0.0162 & 0.3303 \\
NaN & 1 & 2 & 50 & 0.5761 & 0.0155 & 0.3319 \\
5 & 2 & 2 & 50 & 0.5932 & 0.0183 & 0.3519 \\
5 & 2 & 5 & 50 & 0.5932 & 0.0183 & 0.3519 \\
5 & 1 & 2 & 50 & 0.5932 & 0.0182 & 0.3519 \\
5 & 1 & 5 & 50 & 0.5932 & 0.0181 & 0.3519 \\
5 & 2 & 2 & 200 & 0.5934 & 0.0182 & 0.3521 \\
5 & 2 & 5 & 200 & 0.5934 & 0.0182 & 0.3521 \\
5 & 1 & 2 & 200 & 0.5934 & 0.0182 & 0.3522 \\
5 & 1 & 5 & 200 & 0.5935 & 0.0182 & 0.3522 \\
5 & 2 & 2 & 100 & 0.5936 & 0.0182 & 0.3524 \\
5 & 2 & 5 & 100 & 0.5936 & 0.0181 & 0.3524 \\
5 & 1 & 5 & 100 & 0.5937 & 0.0181 & 0.3524 \\
5 & 1 & 2 & 100 & 0.5937 & 0.0181 & 0.3524 \\
5 & 2 & 2 & 1 & 0.6133 & 0.0176 & 0.3761 \\
5 & 2 & 5 & 1 & 0.6133 & 0.0176 & 0.3762 \\
5 & 1 & 2 & 1 & 0.6136 & 0.0176 & 0.3765 \\
5 & 1 & 5 & 1 & 0.6136 & 0.0176 & 0.3766 \\
3 & 2 & 2 & 50 & 0.6146 & 0.0185 & 0.3777 \\
3 & 2 & 5 & 50 & 0.6146 & 0.0185 & 0.3777 \\
3 & 1 & 5 & 50 & 0.6146 & 0.0185 & 0.3777 \\
3 & 1 & 2 & 50 & 0.6146 & 0.0185 & 0.3777 \\
3 & 2 & 2 & 200 & 0.6146 & 0.0185 & 0.3777 \\
3 & 2 & 5 & 200 & 0.6146 & 0.0185 & 0.3777 \\
3 & 1 & 5 & 200 & 0.6146 & 0.0185 & 0.3777 \\
3 & 1 & 2 & 200 & 0.6146 & 0.0185 & 0.3777 \\
3 & 2 & 5 & 100 & 0.6146 & 0.0185 & 0.3777 \\
3 & 2 & 2 & 100 & 0.6146 & 0.0185 & 0.3777 \\
3 & 1 & 2 & 100 & 0.6146 & 0.0185 & 0.3777 \\
3 & 1 & 5 & 100 & 0.6146 & 0.0185 & 0.3777 \\
3 & 2 & 5 & 1 & 0.6261 & 0.0178 & 0.3920 \\
3 & 2 & 2 & 1 & 0.6261 & 0.0178 & 0.3920 \\
3 & 1 & 2 & 1 & 0.6261 & 0.0178 & 0.3920 \\
3 & 1 & 5 & 1 & 0.6261 & 0.0178 & 0.3920 \\
1 & 2 & 2 & 100 & 0.6572 & 0.0188 & 0.4319 \\
1 & 2 & 5 & 100 & 0.6572 & 0.0188 & 0.4319 \\
1 & 1 & 5 & 100 & 0.6572 & 0.0188 & 0.4319 \\
1 & 1 & 2 & 100 & 0.6572 & 0.0188 & 0.4319 \\
1 & 1 & 2 & 200 & 0.6572 & 0.0188 & 0.4319 \\
1 & 1 & 5 & 200 & 0.6572 & 0.0188 & 0.4319 \\
1 & 2 & 5 & 200 & 0.6572 & 0.0188 & 0.4319 \\
1 & 2 & 2 & 200 & 0.6572 & 0.0188 & 0.4319 \\
1 & 1 & 5 & 50 & 0.6572 & 0.0188 & 0.4319 \\
1 & 1 & 2 & 50 & 0.6572 & 0.0188 & 0.4319 \\
1 & 2 & 2 & 50 & 0.6572 & 0.0188 & 0.4319 \\
1 & 2 & 5 & 50 & 0.6572 & 0.0188 & 0.4319 \\
1 & 2 & 5 & 1 & 0.6572 & 0.0188 & 0.4319 \\
1 & 2 & 2 & 1 & 0.6572 & 0.0188 & 0.4319 \\
1 & 1 & 2 & 1 & 0.6572 & 0.0188 & 0.4319 \\
1 & 1 & 5 & 1 & 0.6572 & 0.0188 & 0.4319 \\
10 & 1 & 5 & 1 & 0.6595 & 0.0162 & 0.4350 \\
10 & 2 & 5 & 1 & 0.6605 & 0.0199 & 0.4363 \\
10 & 2 & 2 & 1 & 0.6609 & 0.0199 & 0.4367 \\
10 & 1 & 2 & 1 & 0.6612 & 0.0179 & 0.4372 \\
20 & 2 & 5 & 1 & 0.7736 & 0.0164 & 0.5985 \\
20 & 2 & 2 & 1 & 0.7798 & 0.0166 & 0.6081 \\
NaN & 2 & 5 & 1 & 0.7822 & 0.0156 & 0.6119 \\
20 & 1 & 5 & 1 & 0.7831 & 0.0145 & 0.6132 \\
NaN & 2 & 2 & 1 & 0.7876 & 0.0169 & 0.6204 \\
NaN & 1 & 5 & 1 & 0.7949 & 0.0162 & 0.6319 \\
20 & 1 & 2 & 1 & 0.8008 & 0.0159 & 0.6413 \\
NaN & 1 & 2 & 1 & 0.8128 & 0.0203 & 0.6606 \\
\bottomrule
\end{tabular}

\end{adjustbox}
\end{scriptsize}
\caption{GridSearch para Random Forest.}
\label{tab:random-forest}
\end{table}



\subsubsection{Gradient Boosting}


\begin{table}[H]
\centering
\setlength{\tabcolsep}{4pt}
\begin{scriptsize}
\begin{adjustbox}{width=\textwidth}
\begin{tabular}{rrrrrrrr}
\toprule
param\_model\_\_learning\_rate & param\_model\_\_max\_depth & param\_model\_\_min\_samples\_leaf & param\_model\_\_min\_samples\_split & param\_model\_\_n\_estimators & rmse\_mean & rmse\_std & MSE\_mean \\
\midrule
0.0500 & 7 & 2 & 2 & 200 & 0.5666 & 0.0160 & 0.3210 \\
0.0500 & 7 & 1 & 5 & 200 & 0.5668 & 0.0157 & 0.3212 \\
0.0500 & 7 & 2 & 2 & 100 & 0.5669 & 0.0160 & 0.3213 \\
0.0500 & 7 & 2 & 5 & 200 & 0.5670 & 0.0156 & 0.3215 \\
0.0500 & 7 & 1 & 5 & 100 & 0.5673 & 0.0159 & 0.3219 \\
0.0500 & 7 & 1 & 2 & 200 & 0.5675 & 0.0159 & 0.3221 \\
0.1000 & 7 & 2 & 2 & 50 & 0.5675 & 0.0158 & 0.3221 \\
0.0500 & 7 & 2 & 5 & 100 & 0.5677 & 0.0162 & 0.3223 \\
0.0500 & 7 & 1 & 2 & 100 & 0.5680 & 0.0161 & 0.3226 \\
0.1000 & 7 & 2 & 5 & 50 & 0.5681 & 0.0167 & 0.3228 \\
0.1000 & 7 & 2 & 2 & 100 & 0.5682 & 0.0154 & 0.3228 \\
0.1000 & 5 & 2 & 2 & 100 & 0.5682 & 0.0151 & 0.3229 \\
0.1000 & 5 & 2 & 2 & 200 & 0.5684 & 0.0158 & 0.3231 \\
0.1000 & 7 & 2 & 5 & 100 & 0.5685 & 0.0166 & 0.3232 \\
0.0500 & 5 & 1 & 2 & 200 & 0.5686 & 0.0153 & 0.3233 \\
0.1000 & 5 & 1 & 2 & 200 & 0.5687 & 0.0146 & 0.3234 \\
0.1000 & 5 & 1 & 2 & 100 & 0.5687 & 0.0148 & 0.3235 \\
0.0500 & 5 & 1 & 5 & 200 & 0.5688 & 0.0149 & 0.3236 \\
0.1000 & 7 & 1 & 5 & 50 & 0.5689 & 0.0151 & 0.3237 \\
0.0500 & 5 & 2 & 5 & 200 & 0.5690 & 0.0152 & 0.3238 \\
0.0500 & 5 & 2 & 2 & 200 & 0.5692 & 0.0153 & 0.3240 \\
0.1000 & 5 & 1 & 5 & 100 & 0.5692 & 0.0150 & 0.3240 \\
0.1000 & 5 & 2 & 5 & 100 & 0.5692 & 0.0154 & 0.3240 \\
0.1000 & 7 & 1 & 5 & 100 & 0.5692 & 0.0151 & 0.3240 \\
0.1000 & 5 & 1 & 5 & 200 & 0.5696 & 0.0144 & 0.3245 \\
0.1000 & 7 & 1 & 2 & 50 & 0.5696 & 0.0160 & 0.3245 \\
0.1000 & 5 & 2 & 5 & 200 & 0.5697 & 0.0151 & 0.3246 \\
0.1000 & 7 & 1 & 2 & 100 & 0.5702 & 0.0157 & 0.3252 \\
0.1000 & 5 & 2 & 2 & 50 & 0.5710 & 0.0152 & 0.3261 \\
0.1000 & 7 & 2 & 2 & 200 & 0.5710 & 0.0148 & 0.3261 \\
0.1000 & 5 & 1 & 2 & 50 & 0.5712 & 0.0155 & 0.3263 \\
0.1000 & 5 & 2 & 5 & 50 & 0.5712 & 0.0150 & 0.3263 \\
0.0500 & 5 & 1 & 5 & 100 & 0.5714 & 0.0152 & 0.3265 \\
0.0500 & 5 & 1 & 2 & 100 & 0.5714 & 0.0155 & 0.3265 \\
0.1000 & 5 & 1 & 5 & 50 & 0.5715 & 0.0150 & 0.3266 \\
0.0500 & 5 & 2 & 2 & 100 & 0.5716 & 0.0151 & 0.3267 \\
0.0500 & 5 & 2 & 5 & 100 & 0.5717 & 0.0152 & 0.3268 \\
0.1000 & 3 & 1 & 2 & 200 & 0.5719 & 0.0149 & 0.3270 \\
0.1000 & 3 & 2 & 5 & 200 & 0.5721 & 0.0158 & 0.3273 \\
0.1000 & 3 & 1 & 5 & 200 & 0.5724 & 0.0153 & 0.3276 \\
0.1000 & 7 & 1 & 5 & 200 & 0.5724 & 0.0137 & 0.3276 \\
0.1000 & 3 & 2 & 2 & 200 & 0.5725 & 0.0163 & 0.3277 \\
0.1000 & 7 & 2 & 5 & 200 & 0.5725 & 0.0157 & 0.3278 \\
0.0500 & 7 & 2 & 2 & 50 & 0.5732 & 0.0168 & 0.3286 \\
0.1000 & 7 & 1 & 2 & 200 & 0.5736 & 0.0144 & 0.3290 \\
0.0500 & 7 & 2 & 5 & 50 & 0.5737 & 0.0167 & 0.3291 \\
0.0500 & 7 & 1 & 5 & 50 & 0.5739 & 0.0172 & 0.3294 \\
0.0500 & 7 & 1 & 2 & 50 & 0.5743 & 0.0169 & 0.3298 \\
0.1000 & 3 & 1 & 2 & 100 & 0.5752 & 0.0154 & 0.3309 \\
0.1000 & 3 & 1 & 5 & 100 & 0.5755 & 0.0156 & 0.3311 \\
0.1000 & 3 & 2 & 2 & 100 & 0.5756 & 0.0158 & 0.3313 \\
0.0500 & 3 & 2 & 5 & 200 & 0.5758 & 0.0155 & 0.3315 \\
0.1000 & 3 & 2 & 5 & 100 & 0.5758 & 0.0157 & 0.3315 \\
0.0500 & 3 & 1 & 5 & 200 & 0.5758 & 0.0156 & 0.3315 \\
0.0500 & 3 & 2 & 2 & 200 & 0.5758 & 0.0154 & 0.3316 \\
0.0500 & 3 & 1 & 2 & 200 & 0.5760 & 0.0154 & 0.3318 \\
0.0100 & 7 & 2 & 2 & 200 & 0.5774 & 0.0169 & 0.3333 \\
0.0100 & 7 & 2 & 5 & 200 & 0.5775 & 0.0169 & 0.3335 \\
0.0100 & 7 & 1 & 5 & 200 & 0.5784 & 0.0166 & 0.3346 \\
0.0100 & 7 & 1 & 2 & 200 & 0.5784 & 0.0164 & 0.3346 \\
0.0500 & 5 & 2 & 2 & 50 & 0.5813 & 0.0161 & 0.3379 \\
0.0500 & 5 & 1 & 2 & 50 & 0.5814 & 0.0165 & 0.3380 \\
0.0500 & 5 & 2 & 5 & 50 & 0.5814 & 0.0161 & 0.3380 \\
0.0500 & 5 & 1 & 5 & 50 & 0.5816 & 0.0163 & 0.3383 \\
0.1000 & 3 & 1 & 2 & 50 & 0.5817 & 0.0160 & 0.3384 \\
0.1000 & 3 & 1 & 5 & 50 & 0.5817 & 0.0160 & 0.3384 \\
0.1000 & 3 & 2 & 5 & 50 & 0.5819 & 0.0162 & 0.3386 \\
0.1000 & 3 & 2 & 2 & 50 & 0.5819 & 0.0162 & 0.3386 \\
0.0500 & 3 & 2 & 5 & 100 & 0.5819 & 0.0159 & 0.3386 \\
0.0500 & 3 & 2 & 2 & 100 & 0.5819 & 0.0159 & 0.3386 \\
0.0500 & 3 & 1 & 5 & 100 & 0.5821 & 0.0161 & 0.3389 \\
0.0500 & 3 & 1 & 2 & 100 & 0.5822 & 0.0162 & 0.3389 \\
0.0100 & 5 & 1 & 2 & 200 & 0.5865 & 0.0165 & 0.3440 \\
0.0100 & 5 & 2 & 2 & 200 & 0.5866 & 0.0164 & 0.3441 \\
0.0100 & 5 & 2 & 5 & 200 & 0.5866 & 0.0164 & 0.3441 \\
0.0100 & 5 & 1 & 5 & 200 & 0.5866 & 0.0166 & 0.3441 \\
0.0500 & 3 & 1 & 5 & 50 & 0.5966 & 0.0166 & 0.3559 \\
0.0500 & 3 & 1 & 2 & 50 & 0.5966 & 0.0166 & 0.3559 \\
0.0500 & 3 & 2 & 2 & 50 & 0.5966 & 0.0166 & 0.3559 \\
0.0500 & 3 & 2 & 5 & 50 & 0.5966 & 0.0166 & 0.3559 \\
0.0100 & 3 & 2 & 5 & 200 & 0.6039 & 0.0165 & 0.3647 \\
0.0100 & 3 & 1 & 5 & 200 & 0.6039 & 0.0165 & 0.3647 \\
0.0100 & 3 & 2 & 2 & 200 & 0.6039 & 0.0165 & 0.3647 \\
0.0100 & 3 & 1 & 2 & 200 & 0.6039 & 0.0165 & 0.3647 \\
0.0100 & 7 & 2 & 2 & 100 & 0.6043 & 0.0155 & 0.3651 \\
0.0100 & 7 & 2 & 5 & 100 & 0.6045 & 0.0156 & 0.3654 \\
0.0100 & 7 & 1 & 2 & 100 & 0.6052 & 0.0154 & 0.3663 \\
0.0100 & 7 & 1 & 5 & 100 & 0.6053 & 0.0155 & 0.3664 \\
0.0100 & 5 & 1 & 2 & 100 & 0.6132 & 0.0155 & 0.3760 \\
0.0100 & 5 & 1 & 5 & 100 & 0.6132 & 0.0155 & 0.3760 \\
0.0100 & 5 & 2 & 2 & 100 & 0.6133 & 0.0155 & 0.3761 \\
0.0100 & 5 & 2 & 5 & 100 & 0.6133 & 0.0155 & 0.3761 \\
0.0100 & 3 & 2 & 2 & 100 & 0.6304 & 0.0156 & 0.3974 \\
0.0100 & 3 & 1 & 5 & 100 & 0.6304 & 0.0156 & 0.3974 \\
0.0100 & 3 & 1 & 2 & 100 & 0.6304 & 0.0156 & 0.3974 \\
0.0100 & 3 & 2 & 5 & 100 & 0.6304 & 0.0156 & 0.3974 \\
0.0100 & 7 & 2 & 2 & 50 & 0.6440 & 0.0141 & 0.4148 \\
0.0100 & 7 & 2 & 5 & 50 & 0.6441 & 0.0141 & 0.4148 \\
0.0100 & 7 & 1 & 2 & 50 & 0.6444 & 0.0140 & 0.4153 \\
0.0100 & 7 & 1 & 5 & 50 & 0.6444 & 0.0141 & 0.4153 \\
0.0100 & 5 & 2 & 2 & 50 & 0.6514 & 0.0140 & 0.4244 \\
0.0100 & 5 & 2 & 5 & 50 & 0.6515 & 0.0140 & 0.4244 \\
0.0100 & 5 & 1 & 5 & 50 & 0.6515 & 0.0140 & 0.4244 \\
0.0100 & 5 & 1 & 2 & 50 & 0.6515 & 0.0140 & 0.4245 \\
0.0100 & 3 & 2 & 5 & 50 & 0.6633 & 0.0140 & 0.4399 \\
0.0100 & 3 & 1 & 2 & 50 & 0.6633 & 0.0140 & 0.4399 \\
0.0100 & 3 & 1 & 5 & 50 & 0.6633 & 0.0140 & 0.4399 \\
0.0100 & 3 & 2 & 2 & 50 & 0.6633 & 0.0140 & 0.4399 \\
\bottomrule
\end{tabular}

\end{adjustbox}
\end{scriptsize}
\caption{Resultados del GridSearchCV para Gradient Boosting Regressor.}
\label{tab:gradient-boosting-gridsearch}
\end{table}


\subsubsection{Red Neuronal (MLP)}


\begin{table}[H]
\centering
\setlength{\tabcolsep}{4pt}
\begin{scriptsize}
\begin{adjustbox}{width=\textwidth}
\begin{tabular}{rlrrrr}
\toprule
param\_model\_\_alpha & param\_model\_\_hidden\_layer\_sizes & param\_model\_\_learning\_rate\_init & rmse\_mean & rmse\_std & MSE\_mean \\
\midrule
0.0001 & (128, 64) & 0.0010 & 0.5947 & 0.0182 & 0.3537 \\
0.0010 & (128, 64) & 0.0010 & 0.5947 & 0.0181 & 0.3537 \\
0.0100 & (64,) & 0.0010 & 0.5951 & 0.0170 & 0.3541 \\
0.0100 & (128, 64) & 0.0010 & 0.5951 & 0.0169 & 0.3541 \\
0.0001 & (64,) & 0.0010 & 0.5953 & 0.0171 & 0.3544 \\
0.0010 & (64,) & 0.0010 & 0.5957 & 0.0174 & 0.3549 \\
0.0001 & (64, 32) & 0.0010 & 0.5966 & 0.0161 & 0.3559 \\
0.0010 & (64, 32) & 0.0010 & 0.5976 & 0.0154 & 0.3571 \\
0.0100 & (64, 32) & 0.0010 & 0.5977 & 0.0174 & 0.3572 \\
0.0100 & (64,) & 0.0100 & 0.5999 & 0.0196 & 0.3599 \\
0.0010 & (64,) & 0.0100 & 0.6002 & 0.0191 & 0.3603 \\
0.0001 & (64,) & 0.0100 & 0.6018 & 0.0209 & 0.3622 \\
0.0100 & (128, 64) & 0.0100 & 0.6048 & 0.0153 & 0.3657 \\
0.0100 & (64, 32) & 0.0100 & 0.6052 & 0.0105 & 0.3663 \\
0.0010 & (128, 64) & 0.0100 & 0.6053 & 0.0160 & 0.3663 \\
0.0001 & (128, 64) & 0.0100 & 0.6054 & 0.0156 & 0.3665 \\
0.0001 & (64, 32) & 0.0100 & 0.6058 & 0.0114 & 0.3670 \\
0.0010 & (64, 32) & 0.0100 & 0.6065 & 0.0117 & 0.3678 \\
\bottomrule
\end{tabular}

\end{adjustbox}
\end{scriptsize}
\caption{Resultados del GridSearchCV para el modelo MLPRegressor (Neural Network).}
\label{tab:mlpregressor-gridsearch}
\end{table}



\subsection{Resultados Finales}

A continuación se presentan los mejores hiperparámetros encontrados para cada modelo junto con sus métricas en la partición de validación 70/30:


\begin{table}[H]
\centering
\setlength{\tabcolsep}{4pt}
\begin{scriptsize}
\begin{adjustbox}{width=\textwidth}
\begin{tabular}{lrrrrr}
\toprule
Modelo & rmse\_mean & rmse\_std & MSE\_mean & MAE\_mean & R2\_mean \\
\midrule
GradientBoosting Optimizado & 0.5590 & 0.0086 & 0.3126 & 0.4229 & 0.4194 \\
RandomForest Optimizado & 0.5604 & 0.0098 & 0.3142 & 0.4232 & 0.4165 \\
RandomForest & 0.5624 & 0.0095 & 0.3164 & 0.4249 & 0.4124 \\
GradientBoosting & 0.5719 & 0.0079 & 0.3271 & 0.4366 & 0.3924 \\
NeuralNetwork Optimizado & 0.5849 & 0.0079 & 0.3421 & 0.4465 & 0.3645 \\
Ridge Optimizado & 0.5875 & 0.0067 & 0.3452 & 0.4479 & 0.3587 \\
LinearRegression & 0.5875 & 0.0067 & 0.3452 & 0.4479 & 0.3587 \\
Lasso Optimizado & 0.5884 & 0.0066 & 0.3462 & 0.4494 & 0.3569 \\
DecisionTreeRegressor Optimizado & 0.6025 & 0.0058 & 0.3631 & 0.4593 & 0.3255 \\
NeuralNetwork & 0.6178 & 0.0099 & 0.3818 & 0.4721 & 0.2903 \\
Dummy & 0.7340 & 0.0095 & 0.5388 & 0.5747 & -0.0007 \\
DecisionTree & 0.7955 & 0.0115 & 0.6329 & 0.5993 & -0.1755 \\
\bottomrule
\end{tabular}

\end{adjustbox}
\end{scriptsize}
\caption{Validación cruzada (5-Fold) para todos los modelos optimizados.}
\label{tab:cv-all-models}
\end{table}


\subsubsection{Síntesis de resultados}
\begin{itemize}
    \item \textbf{Hold-out 70/30} (Tabla \ref{tab:model-results-first-barrido}): Random Forest (RMSE 0.5533, MAE 0.4199, R\textsuperscript{2} 0.4281) lidera el desempeño, seguido por Gradient Boosting (RMSE 0.5649, MAE 0.4313, R\textsuperscript{2} 0.4040). Los modelos lineales (Linear Regression, Ridge, Lasso) muestran un rendimiento consistente con un RMSE cercano a 0.586. La Red Neuronal base comienza con un RMSE de 0.6006, mientras que el Árbol de Decisión sin optimizar presenta un sobreajuste severo o mal desempeño (RMSE 0.8009).
    \item \textbf{Validación cruzada (5 pliegues)} (Tabla \ref{tab:cv-all-models}): Se confirma la superioridad de Random Forest Optimizado, logrando el mejor RMSE promedio de 0.5604 ($\pm$ 0.0098) y un R\textsuperscript{2} de 0.4165. Gradient Boosting se mantiene competitivo (RMSE 0.5719), pero por debajo de Random Forest.
    \item \textbf{Mejoras por optimización}: La optimización de hiperparámetros fue crucial para el Árbol de Decisión, reduciendo su RMSE de 0.7955 a 0.6025. La Red Neuronal también mejoró significativamente (de 0.6178 a 0.5849), superando ligeramente a los modelos lineales. Random Forest mostró una mejora marginal con la optimización (0.5624 a 0.5604), lo que indica que su configuración base ya era bastante robusta.
    \item \textbf{Conclusión práctica}: Random Forest Optimizado es el candidato final seleccionado por su consistencia y menor error tanto en validación simple como cruzada. Los modelos lineales sirven como una base sólida y rápida, mientras que las redes neuronales requieren más ajuste para superar a los métodos de ensamble en este dataset tabular.
    \item El valor de \textbf{MSE} obtenido por los modelos en  \textit{Kaggle} fue de  11000.
\end{itemize}
